\chapter*{结论}

本综述主要介绍多铁材料的研究背景与研究现状。多铁材料根据其铁电铁磁性质的来源与相互作用耦合强度可以简单的分为两大类,根据多铁性质的具体原理可以进一步细分。经过多年的发展,多铁材料的研究手段有了突飞猛进的进展,实验方面先进的观察手段可以从更小的时间分辨率与空间分辨率对多铁材料进行物理描述,在理论计算方面第一性原理计算依然是研究设计多铁材料的最佳方法。多铁材料的应用广泛,前途光明。多铁材料在基础研究上对研究铁磁铁电性微观原理有很大帮助。在应用层面电磁耦合的多铁材料可以在存储材料领域有重大价值,可以制造多态存储器、超低能耗存储器等。

多铁材料的研究范围也逐渐宽展到了二维材料领域,二维多铁材料研究的方法与一般的多铁材料相同。自从石墨烯问世以来,以二维材料为代表的低维材料逐渐成为研究热点。二维材料在空间结构上与三维体材料的巨大差别使得二维材料出现了很多新奇的物理特性。二维铁磁、铁电材料十分少见,二维多铁材料的发现之路也十分困难。到目前为止,所发现的性能良好的二维材料主要是第一类多铁材料。目前典型的二维多铁材料主要有过渡金属卤化物、双金属三卤化物等几种。其中基于过渡金属卤化物的二维多铁材料材料是通过对原有的二维铁磁材料进行电子掺杂,是材料内部同时出现电荷排序与轨道排序,进而产生电极化,使铁磁材料产生铁电性。而双金属三卤化物类材料则是具有天然存在的铁电性,通过两种铁电相之间晶体结构的差异使铁磁性在铁磁相与反铁磁相控制磁场,这种创新性的电磁耦合方式使得材料的电磁耦合强度接近第二类多铁材料,但实现通过磁场控制电场依然十分困难。

在二维多铁材料研究领域,第一性原理计算是研究的主流方法。基于密度泛函理论的理论计算工具在研究二维多铁材料上仍然适用,一些分子动力学方法也被应用在二维多铁材料某些性质的研究之中。到目前为止,寻找一种在室温下稳定的有着强电磁耦合的多铁材料仍然是二维多铁材料材料研究领域的首要目标。