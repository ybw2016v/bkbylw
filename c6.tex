\chapter*{附录}

\section*{译文}

多铁材料的演化

探索使原子或分子的磁矩自发统一取向用以产生磁铁(更准确地说是铁磁体)的方法已经持续了两千五百多年,而电偶极矩也可以自发有序排列仅仅在一个世纪前才被发现。电偶极矩自发有序排列这一现象被命名为铁电性是因为出现了类似于铁磁性的在两个稳定态之间切换的磁滞回线,而不是含有铁元素。尽管铁磁性与铁电性的原理与产生机制有着巨大的差异,将这两种性质结合起来,从而创造出兼具两种性质的多铁材料的尝试一直在进行中。多铁材料吸引人们的兴趣主要有两个原因。一方面多铁材料使得同时利用两种性质(铁磁性、铁电性)的功能成为可能,例如磁性状态与铁电状态可以被联合应用以建立起一个四态存储器。另一方面,铁磁性与铁电性的耦合可能会产生在原来两种状态单独存在时所不具备的新的性质与功能。利用电场取代磁场来控制磁性是多铁材料所具备的优点之一。在对磁头进行数据读写的过程中,如果使用一个电亚脉冲来代替用来翻转磁性比特的电流与磁场,电流热效应产生的废热与生成电流与磁场所浪费的时间就可以被避免。多铁材料的应用会产生出速度更快、体积更小、更加节能省电的存储技术。

多铁材料的研究覆盖了从基础学科到工程应用的各个研究领域。除此之外,多铁材料的研究也对相邻领域的研究有着影响,例如复杂的铁磁性与铁电性,氧化物的异质结构与界面,还有一些看起来更加遥远学科,比如宇宙学。在这篇综述之中,我们概括总结了多铁材料在不同领域范围内的曲折发展,并讨论了未来的趋势与挑战。希望获得更全面或更具备技术性报道的读者可参考更广泛的一般性综述2、3,或参考在后续章节中重点介绍的特定方面的综述。

我们首先简单介绍一下多铁材料的早起发展历史。意识到在一些重要类别的材料之中,磁和电长程有序的相互竞争是将多铁材料历史研究与现代研究区分开的里程碑。我们继续总结允许铁磁性与铁电性铁磁性共存的机制,并评估诱导产生电磁耦合的潜在能力。然后,我们仔细研究异质结构,特别是将多铁材料实用化的引入其他性质与功能的界面。磁畴与畴壁也被讨论:在多铁材料中,电磁长程有序非任何类型的耦合都起源于电畴磁畴之间的耦合。然后,我们进一步研究了多铁材料的非平衡动力学,因为考虑到多铁材料的研究重点是电场对磁序的操纵,所以理解控制磁电耦合的过程和时间尺度非常重要。理解多铁材料中电与磁的耦合状态的重要进展通常是通过考虑对称性引入的,对于这个原因我们注意到了多铁材料的基本的对称性。这些材料许多不寻常的性质都与来自于同时存在的铁磁铁电的长程有序的非传统的对称性有关。相反的,有很多化合物展示出了与多铁材料相同的对称性,而且材料的性质也允许这种对称性,但并没有多铁有序性。这些系统也值得与多铁材料相同的探索,我们将在受多铁性存在的显著影响的领域这一章节中讨论,尽管多铁性在它自身中起的作用很小。最后,我们识别出了将继续影响多铁性研究的主要问题与挑战。基础研究以及获得可用多铁器件的努力都在本综述的覆盖范围内。

将铁电与磁有序结合在一个单一的化合物的首次尝试是在前苏联中进行的。在1958年Smolenskii和Ioffe建议在铁电钙钛矿中引入磁性离子,以在不丧失铁电有序性的情况下,形成具有磁性长程有序的固溶体。然而研究最为深入的化合物是硼化物,在其中观察到明显的线性的通过电场或磁场滞回切换多铁畴的磁电效应。

现在的多铁材料的繁荣发展的两个先兆值得注意。第一是在1978年Newnham和共同合作者报告了Cr2BeO4中的螺旋形磁矩打破了空间反演对称,以及由此引起的电极化。在他们的分析之中作者预言了新型的磁驱动的铁电体的许多背后的物理现象,这一物理现象在后来将被应用在获得强电磁相互作用的多铁材料。第二,在1993年,在当时多铁材料研究的基础上,许多现象、系统和专有名词包括多铁材料自身,都是在一个关于电磁现象的会议上形成的,即使在今天,这个会议的会议记录哦也是一本引人入胜的读物。

在2000年,Spaldin重新审视了Smolenskii 和 Ioffe最初的想法,并解释了铁电序和磁电序在钙钛矿晶体结构中相互阻碍的技术上的原因。在钙钛矿结构之中,产生铁电性的原因是临近粒子的电子云杂化,支持偏离中心的离子,这种铁电被称为位移铁电,在外壳3d轨道是空着的时候能量上更加有利。相反的,磁性过渡金属的磁有序需要部分填充3d轨道,这样一来,铁磁性与铁电性的矛盾就相当明显了。这一认识引发了对铁电性由其他与磁有序相容或不具有钙钛矿结构的非色散机制驱动的材料的深入研究。

在上述对钙钛矿型多铁材料的研究之后,寻找允许铁电序和磁序共存的非色散型铁电材料成为多铁材料领域的主要研究热点。根据导致多铁性产生的原因不同可以将多铁材料分为四类。

铁电性可以被孤对电子、几何效应、电荷排序或磁性驱动。在前三个类型之中,铁磁性与铁电性是相互独立的,被成为第一类多铁材料。剩下的一个类型之中,铁电性与铁磁性是同源产生的,这种多铁材料是第二类多铁材料。
\section*{译文原文}
The spontaneous, uniform orientation of atomic or molecular magnetic moments to generate what is colloquially called a magnet (more correctly, a ferromagnet) has been explored for more than 2,500 years. Barely a century ago, it was discovered that spontaneous ordering of electric dipole moments can occur as well1. This phenomenon was named ferroelectricity because of the analogies to ferromagnetism, such as the hysteretic switching between two stable states in an external field. Although the technological merits of ferromagnetism and ferroelectricity are quite different, attempts were made to combine them in the same phase of a mat­erial to create a so­called multiferroic material (BOX 1). Multiferroic materials are interesting mainly for two reasons. On the one hand, they make it possible to exploit the functionalities of both orders; for example, a magnetic bit may be complemented by an electric bit to establish a four­state memory element. On the other hand, a coupling between the ferromagnetic and the ferroelectric states might induce novel functionalities not present in either state alone. The control of the magnetic properties by electric fields instead of magnetic fields is an example of the advantages that multiferroic materials can offer. In the reading and writing of a magnetic bit, if a voltage pulse can be used instead of a magnetic­field­generating electric current, the waste heat and relatively long build­up time associated with electric currents are avoided. Multiferroics may thus lead to faster, smaller, more energy­efficient data­storage technologies.

The field of multiferroics covers aspects ranging from technological applications to abstract problems of fundamental research. In addition, the study of multiferroics increasingly influences neighbouring research areas, such as complex magnetism and ferroelectricity, oxide heterostructures and interfaces, and also seemingly remote subjects such as cosmology. In this Review, we give an overview of the twists and turns in the development of the diverse field of multiferroics, and we discuss the trends and challenges that will define its future. Readers looking for a more comprehensive or more technical coverage are referred to more extensive general reviews2,3, or to reviews on particular aspects that are highlighted in further sections.

We begin with a brief survey of the early days of multi­ferroics. The realization that in some important classes of materials, magnetic and electric long­range order compete with each other4 may be regarded as the milestone separating historical from contemporary research in this field. We continue with an overview of mechanisms permittingthe coexistence of magnetic and ferroelectric order, and evaluate their potential for inducing pronounced magnetoelectric coupling effects (BOX 1). We then scrutinize heterostructures and, in particular, interfaces that introduce additional functionalities, bringing multi ferroics closer to device applications. Domains and domain walls are also discussed: any type of coupling between magnetic and electric long­range order in a multiferroic material has its roots in the coupling between the magnetic and electric domains. We then have a closer look at the non­equilibrium dynamics of multiferroic materials, because, considering that the focus in multi ferroics is on the manipulation of the magnetic order by electric fields, it is very important to understand the processes and timescales governing the magnetoelectric coupling. Important progress in the understanding of the coupling of magnetic and electric states has often been introduced by symmetry considerations; for this reason, we pay some attention to fundamental symmetry properties of multiferroic materials. Many of the unusual properties of these materials are linked to the unconventional symmetry resulting from the simultaneous presence of magnetic and electric long­range order. In turn, there are compounds displaying the symmetry of multiferroic materials, and the material properties permitted by this symmetry, but no multiferroic order. These systems are equally worth exploring, which we cover in the section devoted to research areas that were significantly influenced by the existence of multiferroics, even though multiferroicity in itself has a minor role in them. Finally, we identify major questions and challenges that will continue to influence research in multiferroics. Fundamental aspects, as well as the efforts to obtain working multiferroic devices, are covered in this Review.

Efforts to combine magnetic and ferroelectric order in a single compound were first undertaken in the former Soviet Union. In 1958, Smolenskii and Ioffe5 suggested the introduction of magnetic ions into ferroelectric perovskites to create solid solutions hosting magnetic long­range order without losing the ferroelectric order. However, the most intensely investigated compounds were  boracites,  such  as  Ni3B7O13I,  in  which  a  pronounced linear magnetoelectric effect with hysteretic switching of multiferroic domains by electric or magnetic fields was observed6. The experimental, theoretical and applied achievements of these early days of the field are summarized in REFS 7,8.

Two precursurs of the present boom in multiferroics are  noteworthy.  First,  in  1978,  Newnham  and  co­workers9  reported  that  a  spiral­like  arrangement  of  magnetic moments in Cr2BeO4 breaks spatial inversion symmetry, as does the electric polarization that is thereby induced. In their analysis, the authors foreshadowed much of the physics behind a new type of magnetically driven (improper) ferroelectricity that, a generation later, would be exploited to obtain multiferroics with strong magnetoelectric interactions. Second, in 1993, many of the phenomena, systems and concepts at the base of contemporary multiferroics research, including the invention of the term multiferroicitself, were formulated at a conference on magnetolelectric phenomena — even today, its proceedings are a fascinating read.

In  2000,  Spaldin  revisited  the  original  idea  of  Smolenskii and Ioffe and explained the reason why the ferroelectric and magnetic orders obstruct each other in crystals with the versatile perovskite structure that is technologically relevant4. In perovskites, the ferro­electric state emerges because the electron clouds of neighbouring  ions  hybridize,  which  supports  off­centred ions; this type of ferroelectricity is called displacive ferroelectricity and is particularly energetically favourable when the 3d shell is empty. By contrast, magnetic transition­metal ordering requires partially filled 3d shells — the contradiction is obvious. This realization triggered an intensive search for materials in which ferroelectricity is driven by other, nondisplacive mechanisms that are compatible with magnetic order or that do not have the perovskite structure. The first breakthrough was the discovery of pronounced magnetoelectric interactions in hexagonal (h­) YMnO3(REF. 11), orthorhombic (o­) TbMnO3(REF. 12)  and  TbMn2O5(REF. 13). In these last two materials, the magnetoelectric interaction originates from non­centrosymmetric spin textures that induce a magnetically controllable electric polarization. Contrary to the earlier work on Cr2BeO4(REF. 9), these discoveries inspired the concerted action of different communities — materials science, condensed matter physics and materials theory — that resulted in an impressive expansion of the field of multiferroics.

Following the aforementioned study on perovskite multi­ferroic materials, the search for ferroelectric materials of the nondisplacive type that permit the coexistence of ferroelectric and magnetic order became the main focus of the field. We distinguish four main classes of these materials on the basis of the mechanisms inducing the multiferroicity (FIG. 1). Other reviews present detailed surveys of these classes14,15 and their distinction by diffraction techniques1

Ferroelectricity may be driven by electronic lone pairs, geometric effects, charge order or magnetism. In the first three classes, the magnetic and ferroelectric orders occur independently, and the multiferroic material is denoted as type I. In the last class, the ferroelectric and magnetic transitions emerge jointly, in which case the multiferroic material is type II.
