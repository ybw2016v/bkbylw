\chapter*{摘要}

    多铁材料是指在一个单一的相之中存在一个以上的铁性质的材料,铁性质是指例如铁磁、铁电、铁弹性等性质,这些铁性质之中,最具备理论与应用价值的是铁电与铁磁这两个重要的经典性质,也是目前凝聚态物理研究的热点,目前通常指代将铁磁性行为与铁电相结合的电磁材料。
    多铁材料的研究范围也逐渐宽展到了二维材料领域,二维多铁材料研究的方法与一般的多铁材料相同。二维材料在空间结构上与三维体材料的巨大差别使得二维材料出现了很多新奇的物理特性。二维铁磁、铁电材料十分少见,二维多铁材料的发现之路也十分困难。到目前为止,所发现的性能良好的二维材料主要是第一类多铁材料。目前密度泛函理论仍然是研究二维多铁材料的最佳工具,但一些新方法,如机器学习等也开始逐渐应用到二维多铁材料的研究之中。
    
    \textbf{关键词:} 多铁材料\ 二维材料\ 第一性原理计算\ 密度泛函理论\ 铁磁\ 铁电
    
\chapter*{Abstract}

    Multiferroic materials is a kind of new materials which exhibit more than one ferroic order , such as ferromagnetism, ferroelectricity, ferroelasticity or ferrotoroidicity in a sigle phase. The most significant ferroics in multiferroics are ferroelectricity and ferromagnetism , which play an important role in the eveluation of technical of data storage . The multiferroics with both ferroelectricity and ferromagnetism ,which also called the magnetoelectric multiferroic is going to be one of the centers of the novel material research. With the development of 2D materials , the fields of multiferroic expand from 3D materials to 2D materials . It is the great differences between 3D and 2D materials who has a very small scale in one dimension which is near the microcosmic basic particals that makes it possible for some novel phenomenon which usually exists in microcosmic world which is under the  contral of quantum mechanics to exhibit in the macroscopic world. The general methods used in the 2D multiferroics research is very similar to the methods in 3D multiferroics research . The 2D multiferroics that is discovered by experiment observation or theoretical prediction up to now  is rare , which means the way to find a 2D multiferroics that can be put into use is very hard. The most of multiferroics that discovered up to now is the type one multiferroics , which means the ferroelectricity and ferromagnetism are drived independently. The best way to deal with the multiferroics in theoretical methods is the first principle calculation which is generally used in materials fields , but some other new methods such as machine learning , neural networks is developing well and start being put into use in many fields.

 
    \textbf{keywords:} multiferroics , 2D-materials , first principle calculation ,  DFT , ferroelectricity ,ferromagnetism    
    
    
